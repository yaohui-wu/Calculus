\subsection{The Limit of a Function}

\subsubsection{Functions}
A function \(f:X\mapsto Y\) is a rule that assigns each element in set \(X\)
to exactly one element \(y=f(x)\) in set \(Y\).
\begin{definition}
    A \textbf{function} \(f\) is a binary relation \(R\) with domain \(X\) and
    codomain \(Y\) that satisfies:
    \begin{itemize}
        \item \(R\) is a subset of the Cartesian product of \(X\) and \(Y\).
        \[R\subset\{(x,y)\mid x\in X,y\in Y\}\]
        \item For every \(x\) in \(X\), there exists a \(y=f(x)\) such that
        \((x,y)\) is in \(R\).
        \[\forall x\in X,\exists y=f(x)\in Y,(x,y)\in R\]
        \item If \((x,y)\) and \((x,z)\) are in \(R\), then \(y=z\).
        \[(x,y)\in R \wedge (x,z)\in R \implies y=z\]
    \end{itemize}
\end{definition}

\subsubsection{Intuitive Definition of a Limit}
Suppose a function \(f(x)\) is defined on some open interval that contains
\(a\), except possibly at \(a\) itself.
\begin{definition}
    The \textbf{limit} of \(f(x)\) as \(x\) approaches \(a\) equals \(L\) if
    we can make \(f(x)\) arbitrarily close to \(L\) by taking \(x\)
    sufficiently close to \(a\) but not equal to \(a\) from left and right.
    \[\lim_{x\to a}f(x)=L\]
\end{definition}
\begin{definition}
    The limit of \(f(x)\) as \(x\) approaches \(a\) from the left equals \(L\)
    if we can make \(f(x)\) arbitrarily close to \(L\) by taking \(x\)
    sufficiently close to \(a\) where \(x<a\).
    \[\lim_{x\to a^-}f(x)=L\]
\end{definition}
\begin{definition}
    The limit of \(f(x)\) as \(x\) approaches \(a\) from the right equals
    \(L\) if we can make \(f(x)\) arbitrarily close to \(L\) by taking \(x\)
    sufficiently close to \(a\) where \(x>a\).
    \[\lim_{x\to a^+}f(x)=L\]
\end{definition}
The limit \textbf{exists} if the limit of \(f(x)\) as \(x\) approaches \(a\)
equals \(L\), otherwise the limit \textbf{does not exist}.
\[\lim_{x\to a}f(x)=L\ \text{if} \lim_{x\to a^-}f(x)=\lim_{x\to a^+}f(x)=L\]