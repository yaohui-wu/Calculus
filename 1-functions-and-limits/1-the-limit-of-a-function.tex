\subsection{The Limit of a Function}
A function \(f:X\mapsto Y\) is a rule that assigns each element in set \(X\) to exactly
one element \(y=f(x)\) in set \(Y\).
\begin{definition}
    A \textbf{function} \(f\) is a binary relation \(R\) with domain \(X\) and
    codomain \(Y\) that satisfies:
    \begin{itemize}
        \item \(R\) is a subset of the Cartesian product of \(X\) and \(Y\).
        \[R\subset\{(x,y)\mid x\in X,y\in Y\}\]
        \item For every \(x\) in \(X\), there exists a \(y=f(x)\) such that
        \((x,y)\) is in \(R\).
        \[\forall x\in X,\exists y=f(x)\in Y,(x,y)\in R\]
        \item If \((x,y)\) and \((x,z)\) are in \(R\), then \(y=z\).
        \[(x,y)\in R \wedge (x,z)\in R \implies y=z\]
    \end{itemize}
\end{definition}