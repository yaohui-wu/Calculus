\subsection{The Precise Definition of a Limit}

\subsubsection{Epsilon-Delta Definition of a Limit}
\textbf{Augustin-Louis Cauchy} (1789--1857) and \textbf{Karl Weierstrass}
(1815--1897) formalized a rigorous definition of a limit.
\begin{definition}
    \[\lim_{x\to a}f(x)=L\] if for every number \(\epsilon>0\), there is a
    number \(\delta>0\) such that
    \[0<|x-a|<\delta\implies|f(x)-L|<\epsilon\qedhere\]
\end{definition}
\begin{definition}
    \[\lim_{x\to a^-}f(x)=L\] if for every number \(\epsilon>0\), there is a
    number \(\delta>0\) such that
    \[a-\delta<x<a\implies|f(x)-L|<\epsilon\qedhere\]
\end{definition}
\begin{definition}
    \[\lim_{x\to a^+}f(x)=L\] if for every number \(\epsilon>0\), there is a
    number \(\delta>0\) such that
    \[a<x<a+\delta\implies|f(x)-L|<\epsilon\qedhere\]
\end{definition}
\begin{problem}
    Prove that \[\lim_{x\to 3}(4x-5)=7\]
\end{problem}
\begin{solution}
    Let \(\epsilon>0\) be given, we want to find a number \(\delta>0\) such
    that \[0<|x-3|<\delta \implies|(4x-5)-7|<\epsilon\]
    We simplify to get \(|(4x-5)-7|=|4x-12|=4|x-3|\) so we have
    \[4|x-3|<\epsilon\iff|x-3|<\frac{\epsilon}{4}\]
    Let \(\delta=\epsilon/4\), we have
    \[0<|x-3|<\frac{\epsilon}{4}\implies4|x-3|<\epsilon\implies
    |(4x-5)-7|<\epsilon\]
    Therefore, by the definition of a limit, it is proved that
    \[\lim_{x\to 3}(4x-5)=7\qedhere\]
\end{solution}
\begin{problem}
    Prove that \[\lim_{x\to 3}x^2=9\]
\end{problem}
\begin{solution}
    Let \(\epsilon>0\) be given, we want to find a number \(\delta>0\) such
    that \[0<|x-3|<\delta\implies|x^2-9|<\epsilon\]
    We simplify to get \[|x^2-9|=|x+3|\,|x-3|<\epsilon\]
    Let \(C\) be a positive constant such that
    \[|x+3|\,|x-3|<C\,|x-3|<\epsilon \iff |x-3|<\frac{\epsilon}{C}\]
    Since we are interested only in values of \(x\) that are close to 3, it is
    reasonable to assume that \(|x-3|<1\) such that \(|x+3|<7\) so \(C=7\).
    Let \(\delta=\min\{1,\epsilon/7\}\), we have
    \begin{gather*}
        0<|x-3|<1\iff|x+3|<7 \\ 0<|x-3|<\frac{\epsilon}{7}\iff
        7\,|x-3|<\epsilon \\ |x+3|\,|x-3|<7\,|x-3|<\epsilon\implies
        |x^2-9|<\epsilon
    \end{gather*}
    Therefore, it is proved that \[\lim_{x\to 3}x^2=9\qedhere\]
\end{solution}
\begin{problem}
    Prove that \[\lim_{x\to 0^+}\sqrt{x}=0\]
\end{problem}
\begin{solution}
    Let \(\epsilon>0\) be given, we want to find a number \(\delta>0\) such
    that \[0<x<\delta\implies|\sqrt{x}-0|<\epsilon\]
    We simplify to get \(\sqrt{x}<\epsilon\iff x<\epsilon^2\).
    Let \(\delta=\epsilon^2\), we have
    \[0<x<\epsilon^2\implies|\sqrt{x}-0|<\epsilon\]
    Therefore, it is proved that \[\lim_{x\to 0^+}\sqrt{x}=0\qedhere\]
\end{solution}