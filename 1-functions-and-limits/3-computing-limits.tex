\subsection{Computing Limits}

\subsubsection{Limit Laws}
Suppose that \(c\) is a constant and the limits
\begin{gather*}
    \lim_{x\to a}f(x)=L,\,\lim_{x\to a}g(x)=M
\end{gather*}
exist.
We have the following properties of limits called the \textbf{limit laws} to
compute limits.
\begin{theorem}
    \[\lim_{x\to a}c=c\]
\end{theorem}
\begin{proof}
    Let \(\epsilon>0\) be given, we want to find a number \(\delta>0\) such
    that \[0<|x-a|<\delta\implies|c-c|<\epsilon\]
    We have \(|c-c|=0<\epsilon\) so the trivial inequality is always true for
    any number \(\delta>0\).
\end{proof}
\begin{theorem}
    \[\lim_{x\to a}x=a\]
\end{theorem}
\begin{proof}
    Let \(\epsilon>0\) be given, we want to find a number \(\delta>0\) such
    that \[0<|x-a|<\delta\implies|x-a|<\epsilon\]
    Let \(\delta=\epsilon\) so we have
    \[0<|x-a|<\delta=\epsilon\implies|x-a|<\epsilon\]
\end{proof}
\begin{theorem}[Sum and Difference Law]
    The limit of a sum or difference is the sum or difference of the limits.
    \[\lim_{x\to a}[f(x)\pm g(x)]=\lim_{x\to a}f(x)\pm\lim_{x\to a}g(x)=L\pm M\]
\end{theorem}
\begin{proof}
    First we prove the sum law.
    Let \(\epsilon>0\) be given, we want to find a number \(\delta>0\) such
    that \[0<|x-a|<\delta\implies|f(x)+g(x)-(L+M)|<\epsilon\]
    By the \textbf{triangle inequality} \(|a+b|\leq|a|+|b|\) we get
    \[|f(x)+g(x)-(L+M)|=|f(x)-L+g(x)-M|\leq|f(x)-L|+|g(x)-M|\]
\end{proof}
\begin{theorem}[Constant Multiple Law]
    The limit of a constant times a function is the constant times the limit
    of the function.
    \[\lim_{x\to a}[c\,f(x)]=c\lim_{x\to a}f(x)=cL\]
\end{theorem}
\begin{proof}
    Let \(\epsilon>0\) be given, we want to find a number \(\delta>0\) such
    that \[0<|x-a|<\delta\implies|f(x)-c\lim_{x\to a}f(x)|<\epsilon\]
\end{proof}
\begin{theorem}[Product Law]
    The limit of a product is the product of the limits.
    \[\lim_{x\to a}[f(x)g(x)]=\lim_{x\to a}f(x)\cdot\lim_{x\to a}g(x)=L\cdot M\]
\end{theorem}
\begin{theorem}[Quotient Law]
    The limit of a quotient is the quotient of the limits (provided that the
    limit of the denominator is not 0).
    \[\lim_{x\to a}\frac{f(x)}{g(x)}=\frac{\lim_{x\to a}f(x)}{\lim_{x\to a}g(x)}
    =\frac{L}{M}\iff\lim_{x\to a}g(x)=M\neq0\]
\end{theorem}
\begin{theorem}[Power Law]
    \[\lim_{x\to a}[f(x)]^n = [\lim_{x\to a}f(x)]^n\]
\end{theorem}
\begin{theorem}[Root Law]
    \[\lim_{x\to a}\sqrt[n]{f(x)} = \sqrt[n]{\lim_{x\to a}f(x)}\]
\end{theorem}