\subsection{Computing Limits}

\subsubsection{Limit Laws}
Suppose that \(c\) is a constant and the limits
\begin{align*}
    \lim_{x\to a}f(x)&=L & \lim_{x\to a}g(x)&=M
\end{align*}
exists.
We have the following properties of limits called the \textbf{limit laws} to
compute limits.
\begin{theorem}
    \[\lim_{x\to a}c=c\]
\end{theorem}
\begin{proof}
    Let \(\epsilon>0\) be given, we want to find a number \(\delta>0\) such
    that \[0<|x-a|<\delta \implies |c-c|<\epsilon\]
    Since \(|c-c|=0<\epsilon\) so the trivial inequality is always true.
    Let \(\delta>0\) be any number and the proof is obvious.
\end{proof}
\begin{theorem}
    \[\lim_{x\to a}x=a\]
\end{theorem}
\begin{proof}
    Let \(\epsilon>0\) be given, we want to find a number \(\delta>0\) such
    that \[0<|x-a|<\delta \implies |x-a|<\epsilon\]
    Let \(\delta=\epsilon\) such that \[0<|x-a|<\delta=\epsilon\]
    and the proof is trivial.
\end{proof}
\begin{theorem}[Constant Multiple Law]
    The limit of a constant times a function is the constant times the limit
    of the function.
    \[\lim_{x\to a}[c\,f(x)]=c\lim_{x\to a}f(x)\]
\end{theorem}
\begin{theorem}[Sum and Difference Law]
    The limit of a sum or difference is the sum or difference of the limits.
    \[\lim_{x\to a}[f(x)\pm g(x)]=L\pm M\]
\end{theorem}
\begin{theorem}[Product Law]
    The limit of a product is the product of the limits.
    \[\lim_{x\to a}[f(x)g(x)]=\lim_{x\to a}f(x)\cdot\lim_{x\to a}g(x)\]
\end{theorem}
\begin{theorem}[Quotient Law]
    The limit of a quotient is the quotient of the limits (provided that the
    limit of the denominator is not 0).
    \[\lim_{x\to a}\frac{f(x)}{g(x)}=\frac{\lim_{x\to a}f(x)}{\lim_{x\to a}g(x)}
    \iff \lim_{x\to a}g(x)\neq0\]
\end{theorem}
\begin{theorem}[Power Law]
    \[\lim_{x\to a}[f(x)]^n = [\lim_{x\to a}f(x)]^n\]
\end{theorem}
\begin{theorem}[Root Law]
    \[\lim_{x\to a}\sqrt[n]{f(x)} = \sqrt[n]{\lim_{x\to a}f(x)}\]
\end{theorem}