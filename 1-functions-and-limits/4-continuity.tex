\subsection{Continuity}
Let \(f(x)\) be a function and the number \(a\) is in the domain of \(f\) so
\(f(a)\) is defined.
If the limit exists, then we have the following definition.
\begin{definition}
    A function \(f\) is \textbf{continuous} at the number \(a\) if
    \[\lim_{x\to a}f(x)=f(a)\]
    A function \(f\) is continuous from the left at \(a\) if the left-hand limit
    equals \(f(a)\) and it is continuous from the right at \(a\) if the right-hand
    limit equals \(f(a)\).
    A function \(f\) is continuous on an interval if it is continuous
    at every number in the interval.
    If \(f\) is not continuous at \(a\), then it is a discontinuous function at \(a\).
\end{definition}
\begin{theorem}
    If \(f\) and \(g\) are continuous functions at \(a\) and \(c\) is a constant,
    then the following functions are also continuous at \(a\). \\
    \begin{itemize*}
        \item \(f+g\) \item \(f-g\) \item \(cf\) \item \(f\cdot g\)
        \item \(\dfrac{f}{g}\iff g(x)\neq 0\)
    \end{itemize*}
\end{theorem}
\begin{theorem}
    Let \(P(x)\) be any polynomial, then \(P(x)\) is continuous on
    \(\R=(-\infty,\infty)\).
\end{theorem}
\begin{proof}
    A polynomial \(P(x)\) is a function of the form
    \[a_nx^n+a_{n-1}x^{n-1}+\dotsb +a_{2}x^{2}+a_1x+a_0\]
    where the coefficients \(a_i\) are constants.
    \(P(x)\) is the sum of power functions with a constant multiple and
    therefore it is continuous.
\end{proof}
\begin{theorem}
    Let \(f\) be any rational function, then \(f\) is continuous on its domain.
\end{theorem}
\begin{proof}
    A rational function \(f\) is a function of the form
    \[f(x)=\frac{P(x)}{Q(x)}\] where \(P\) and \(Q\) are polynomials.
    We know that polynomials are continuous so a rational function is
    continuous on its domain.
\end{proof}
\begin{theorem}
    Polynomials, rational functions, root functions, trigonometric
    functions, inverse trigonometric functions, logarithmic functions, and
    exponential functions are continuous on their domain.
\end{theorem}
\begin{theorem}
    If \(f\) is a one-to-one continuous function defined on an interval
    \([a,b]\), then its inverse function \(f^{-1}\) is also continuous.
\end{theorem}
\begin{theorem}
    If \(f\) is continuous at \(b\) and \(\lim_{x\to a}g(x)=b\), then
    the limit of the composite function \(f\circ g\) is
    \[\lim_{x\to a}f(g(x))=f\left(\lim_{x\to a}g(x)\right)=f(b)\]
\end{theorem}
\begin{proof}
    Let \(\epsilon>0\) be given, we want to find \(\delta>0\) such that
    \[0<|x-a|<\delta\implies|f(g(x))-f(b)|<\epsilon\]
    Since \(f\) is continuous at \(b\), then we have \(\lim_{y\to b}f(y)=f(b)\).
    There exists \(\delta_1>0\) such that
    \[0<|y-b|<\delta_1\implies|f(y)-f(b)|<\epsilon\]
    Since \(\lim_{x\to a}g(x)=b\), there exists \(\delta>0\) such that
    \[0<|x-a|<\delta\implies|g(x)-b|<\delta_1\implies|f(g(x))-f(b)|<\epsilon\]
    Therefore, it is proved that
    \[\lim_{x\to a}f(g(x))=f\left(\lim_{x\to a}g(x)\right)=f(b)\qedhere\]
\end{proof}
\begin{theorem}
    If \(g\) is continuous at \(a\) and \(f\) is continuous at \(g(a)\), then
    \(f\circ g\) is continuous at \(a\).
\end{theorem}
\begin{proof}
    Since \(g\) is continuous at \(a\), we have \(\lim_{x\to a}g(x)=g(a)\).
    Since \(f\) is continuous at \(g(a)\), we have
    \[\lim_{x\to a}f(g(x))=f(g(a))\]
    Therefore, \(f(g(x))\) is continuous at \(a\).
\end{proof}
An important property of continuous functions is formulated by the following
theorem proved by \textbf{Bernard Bolzano} (1781-1848).
\begin{theorem}[Intermediate Value Theorem]
    Suppose that \(f\) is continuous on the closed interval \([a,b]\) and let
    \(n\) be any number between \(f(a)\) and \(f(b)\) where \(f(a)\neq f(b)\)
    such that \[\min\{f(a),f(b)\}<n<\max\{f(a),f(b)\}\]
    Then there exists a number \(c\) in the open interval \((a,b)\) such that
    \(f(c)=n\).
\end{theorem}