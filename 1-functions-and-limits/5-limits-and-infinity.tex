\subsection{Limits and Infinity}
\subsubsection{Infinite Limits}
\begin{definition}
    The limit of \(f(x)\) as \(x\) approaches \(a\) is \textbf{infinity} if
    the values of \(f(x)\) can be made arbitrarily large by taking \(x\)
    sufficiently close to \(a\) but not equal to \(a\).
    \[\lim_{x\to a}f(x)=\infty\]
\end{definition}
\begin{definition}
    The limit of \(f(x)\) as \(x\) approaches \(a\) is \textbf{negative infinity}
    if the values of \(f(x)\) can be made arbitrarily small by taking \(x\)
    sufficiently close to \(a\) but not equal to \(a\).
    \[\lim_{x\to a}f(x)=-\infty\]
\end{definition}
Similar definitions can be given for one-sided infinite limits.
\begin{align*}
    &\lim_{x\to a^-}f(x)=\infty&&\lim_{x\to a^+}f(x)=\infty&
    &\lim_{x\to a^-}f(x)=-\infty&&\lim_{x\to a^+}f(x)=-\infty&
\end{align*}
\begin{definition}
    The \textbf{vertical asymptote} of the curve \(y=f(x)\) is the line \(x=a\)
    if one of the infinite limits is infinity or negative infinity.
\end{definition}

\subsubsection{Limits at Infinity}
\begin{definition}
    Let \(f\) be a function defined on some interval \((a,\infty)\).
    The limit of \(f(x)\) as \(x\) approaches infinity is \(L\) if the values
    of \(f(x)\) can be made as close to \(L\) as we like by taking \(x\)
    sufficiently large.
    \[\lim_{x\to \infty}f(x)=L\]
\end{definition}
\begin{definition}
    Let \(f\) be a function defined on some interval \((-\infty,a)\).
    The limit of \(f(x)\) as \(x\) approaches negative infinity is \(L\) if
    the values of \(f(x)\) can be made as close to \(L\) as we like by taking \(x\)
    sufficiently small.
    \[\lim_{x\to -\infty}f(x)=L\]
\end{definition}
\begin{definition}
    The \textbf{horizontal asymptote} of the curve \(y=f(x)\) is the line \(y=L\)
    if one of the limits at infinity is \(L\).
\end{definition}

\subsubsection{Infinite Limits at Infinity}
\begin{definition}
    The values of \(f(x)\) become large as \(x\) becomes large.
    \[\lim_{x\to \infty}f(x)=\infty\]
\end{definition}
Similar definitions can be given for other infinite limits at infinity or
negative infinity.
\begin{align*}
    &\lim_{x\to \infty}f(x)=-\infty&&\lim_{x\to -\infty}f(x)=\infty&
    &\lim_{x\to -\infty}f(x)=-\infty&
\end{align*}

\subsubsection{Precise Definitions}
Let \(f\) be a function defined on some open interval that contains the number
\(a\), except possibly at \(a\) itself.
\begin{definition}
    \[\lim_{x\to a}f(x)=\infty\] if for every \(M>0\), there is
    \(\delta>0\) such that \[0<|x-a|<\delta\implies f(x)>M\]
\end{definition}
\begin{problem}
    Prove that \[\lim_{x\to a}\frac{1}{x^2}=\infty\]
\end{problem}
Let \(f\) be a function defined on some interval \((a,\infty)\).
\begin{definition}
    \[\lim_{x\to \infty}f(x)=L\] if for every \(\epsilon>0\), there is an \(N\)
    such that \[x>N\implies|f(x)-L<\epsilon|\]
\end{definition}
\begin{problem}
    Prove that \[\lim_{x\to \infty}\frac{1}{x}=0\]
\end{problem}
\begin{definition}
    \[\lim_{x\to \infty}f(x)=\infty\] if for every \(M>0\), there is an \(N>0\)
    such that \[x>N\implies f(x)>M\]
\end{definition}
Similar definitions apply for limits involving negative infinity.