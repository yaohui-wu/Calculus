\subsection{Differentiation}
\subsubsection{Differentiation Formulas}
Let \(f(x)\) and \(g(x)\) be differentiable functions, then we have the
following differentiation formulas.
\begin{theorem}
    Let \(f(x)=c\) where \(c\) is a constant, then
    \[\frac{d}{dx}(c)=0\qedhere\]
\end{theorem}
\begin{proof}
    \[f'(x)=\lim_{h\to 0}\frac{c-c}{h}=\lim_{h\to 0}0=0\qedhere\]
\end{proof}
\begin{theorem}[Power Rule]
    \[\frac{d}{dx}(x^n)=nx^{n-1},\quad n\in\R\qedhere\]
\end{theorem}
\begin{proof}
    We prove the power rule for \(n\in\N\).
    \[\frac{d}{dx}(x^n)=\lim_{h\to 0}\frac{(x+h)^n-x^n}{h}\]
    We use the binomial theorem to expand \((x+h)^n\) then we have
    \begin{align*}
        \frac{d}{dx}(x^n)
        &= \lim_{h\to 0}\frac{x^n+nx^{n-1}h+\dots+nxh^{n-1}+h^n-x^n}{h} \\
        &= \lim_{h\to 0}(nx^{n-1}+\dots+nxh^{n-2}+h^{n-1})=nx^{x-1}
    \end{align*}
    because every term has a factor of \(h\) except \(nx^{n-1}\).
\end{proof}
Note the special case when \(n=1\), then we have \[\frac{d}{dx}(x)=1\]
\begin{problem}
    Differentiate \(f(x)=1/x\).
\end{problem}
\begin{solution}
    \[\frac{d}{dx}\left(\frac{1}{x}\right)=\frac{d}{dx}x^{-1}=(-1)x^{-1-1}
    =-x^{-2}=-\frac{1}{x^2}\qedhere\]
\end{solution}
\begin{theorem}[Constant Multiple Rule]
    If \(c\) is a constant, then
    \[\frac{d}{dx}[cf(x)]=c\,\frac{d}{dx}f(x)\qedhere\]
\end{theorem}
\begin{proof}
    \begin{align*}
        \frac{d}{dx}[cf(x)]&= \lim_{h\to 0}\frac{cf(x+h)-cf(x)}{h}
        =\lim_{h\to 0}\left[c\left(\frac{f(x+h)-f(x)}{h}\right)\right] \\
        &= c\left(\lim_{h\to 0}\frac{f(x+h)-f(x)}{h}\right)
        =c\,\frac{d}{dx}f(x)\qedhere
    \end{align*}
\end{proof}
\begin{theorem}[Sum and Difference Rule]
    \[\frac{d}{dx}[f(x)\pm g(x)]
    =\frac{d}{dx}f(x)\pm \frac{d}{dx}g(x)\qedhere\]
\end{theorem}
\begin{proof}
    We prove the sum rule.
    \begin{align*}
        \frac{d}{dx}[f(x)+g(x)]
        &= \lim_{h\to 0}\frac{f(x+h)+g(x+h)-[f(x)+g(x)]}{h} \\
        &= \lim_{h\to 0}\frac{f(x+h)-f(x)+g(x+h)-g(x)}{h} \\
        &= \lim_{h\to 0}\frac{f(x+h)-f(x)}{h}
        +\lim_{h\to 0}\frac{g(x+h)-g(x)}{h} \\
        &= \frac{d}{dx}f(x)+\frac{d}{dx}g(x)
    \end{align*}
    Then we prove the difference rule.
    \begin{align*}
        \frac{d}{dx}[f(x)-g(x)]
        &= \lim_{h\to 0}\frac{f(x+h)-g(x+h)-[f(x)-g(x)]}{h} \\
        &= \lim_{h\to 0}\frac{f(x+h)-f(x)-g(x+h)+g(x)}{h} \\
        &= \lim_{h\to 0}\frac{f(x+h)-f(x)-[g(x+h)-g(x)]}{h} \\
        &= \lim_{h\to 0}\frac{f(x+h)-f(x)}{h}
        -\lim_{h\to 0}\frac{g(x+h)-g(x)}{h} \\
        &= \frac{d}{dx}f(x)-\frac{d}{dx}g(x)\qedhere
    \end{align*}
\end{proof}

\subsubsection{Product and Quotient Rules}
Let \(f(x)\) and \(g(x)\) be differentiable functions, then we have the
product rule by Leibniz and the quotient rule.
\begin{theorem}[Product Rule]
    \[\frac{d}{dx}[f(x)g(x)]=\left[\frac{d}{dx}f(x)\right]g(x)
    +f(x)\left[\frac{d}{dx}g(x)\right]\qedhere\]
\end{theorem}
\begin{proof}
    \begin{align*}
        \frac{d}{dx}[f(x)g(x)]
        &= \lim_{h\to 0}\frac{f(x+h)g(x+h)-f(x)g(x)}{h} \\
        &= \lim_{h\to 0}\frac{f(x+h)g(x+h)-f(x)g(x)+f(x+h)g(x)-f(x+h)g(x)}{h} \\
        &= \lim_{h\to 0}\frac{f(x+h)g(x)-f(x)g(x)+f(x+h)g(x+h)-f(x+h)g(x)}{h} \\
        &= \lim_{h\to 0}\left[\frac{f(x+h)-f(x)}{h}g(x)\right]
        +\lim_{h\to 0}\left[f(x+h)\frac{g(x+h)-g(x)}{h}\right] \\
        &= \left[\lim_{h\to 0}\frac{f(x+h)-f(x)}{h}\right]g(x)
        +\lim_{h\to 0}f(x+h)\cdot\lim_{h\to 0}\frac{g(x+h)-g(x)}{h} \\
        &= \left[\frac{d}{dx}f(x)\right]g(x)
        +f(x)\left[\frac{d}{dx}g(x)\right]\qedhere
    \end{align*}
\end{proof}
\begin{theorem}[Quotient Rule]
    \[\frac{d}{dx}\left[\frac{f(x)}{g(x)}\right]
    =\frac{\left[\cfrac{d}{dx}f(x)\right]g(x)
    -f(x)\left[\cfrac{d}{dx}g(x)\right]}{[g(x)]^2}\qedhere\]
\end{theorem}
\begin{proof}
    \begin{align*}
        \frac{d}{dx}\left[\frac{f(x)}{g(x)}\right]
        &= \lim_{h\to 0}\frac{\cfrac{f(x+h)}{g(x+h)}-\cfrac{f(x)}{g(x)}}{h}
        =\lim_{h\to 0}\frac{f(x+h)g(x)-f(x)g(x+h)}{h\cdot g(x+h)g(x)} \\
        &= \lim_{h\to 0}\frac{f(x+h)g(x)-f(x)g(x)-[f(x)g(x+h)-f(x)g(x)]}{h}
        \cdot\lim_{h\to 0}\frac{1}{g(x+h)g(x)} \\
        &= \left(\lim_{h\to 0}\left[\frac{f(x+h)-f(x)}{h}g(x)\right]
        -\lim_{h\to 0}\left[f(x)\frac{g(x+h)-g(x)}{h}\right]\right)
        \frac{1}{[g(x)]^2} \\
        &= \left[\left(\lim_{h\to 0}\frac{f(x+h)-f(x)}{h}\right)g(x)
        -f(x)\left(\lim_{h\to 0}\frac{g(x+h)-g(x)}{h}\right)\right]
        \frac{1}{[g(x)]^2} \\
        &= \frac{\left[\cfrac{d}{dx}f(x)\right]g(x)
        -f(x)\left[\cfrac{d}{dx}g(x)\right]}{[g(x)]^2}\qedhere
    \end{align*}
\end{proof}

\subsubsection{Trigonometric Functions}
\begin{theorem}
    \[\frac{d}{dx}\sin x=\cos x\qedhere\]
\end{theorem}
\begin{proof}
    We use the \textbf{angle sum identity} of the sine function
    \[\sin(\alpha+\beta)=\sin\alpha\cos\beta+\cos\alpha\sin\beta\]
    then we have
    \[\frac{d}{dx}\sin x=\lim_{h\to 0}\frac{\sin(x+h)-\sin x}{h}
    =\lim_{h\to 0}\frac{\sin x\cos h+\cos x\sin h-\sin x}{h}\]
    Note that we are taking the limit with respect to \(h\) so \(\sin x\) and
    \(\cos x\) are constants then we have
    \begin{align*}
        \frac{d}{dx}\sin x
        &= \lim_{h\to 0}\left[\frac{\sin x(\cos h-1)}{h}
        +\frac{\cos x\sin h}{h}\right]
        =\lim_{h\to 0}\frac{\sin x(\cos h-1)}{h}
        +\lim_{h\to 0}\frac{\cos x\sin h}{h}  \\
        &= \left(\lim_{h\to 0}\sin x\right)
        \left(\lim_{h\to 0}\frac{\cos h-1}{h}\right)
        +\left(\lim_{h\to 0}\cos x\right)
        \left(\lim_{h\to 0}\frac{\sin h}{h}\right)  \\
        &= (\sin x)(0)+(\cos x)(1) =\cos x\qedhere
    \end{align*}
\end{proof}
\begin{theorem}
    \[\frac{d}{dx}\cos x=-\sin x\qedhere\]
\end{theorem}
\begin{proof}
    We use the angle sum identity of the cosine function
    \[\cos(\alpha+\beta)=\cos\alpha\cos\beta-\sin\alpha\sin\beta\]
    then we have
    \begin{align*}
        \frac{d}{dx}\cos x
        &= \lim_{h\to 0}\frac{\cos(x+h)-\cos x}{h}
        =\lim_{h\to 0}\frac{\cos x\cos h-\sin x\sin h-\cos x}{h}  \\
        &= \lim_{h\to 0}\left(\frac{\cos x(\cos h-1)}{h}
        -\frac{\sin x\sin h}{h}\right)  \\
        &= \left(\lim_{h\to 0}\cos x\right)
        \left(\lim_{h\to 0}\frac{\cos h-1}{h}\right)
        -\left(\lim_{h\to 0}\sin x\right)
        \left(\lim_{h\to 0}\frac{\sin h}{h}\right)  \\
        &= (\cos x)(0)-(\sin x)(1)=-\sin x\qedhere
    \end{align*}
\end{proof}
\begin{theorem}
    \[\frac{d}{dx}\tan x=\sec^2 x\qedhere\]
\end{theorem}
\begin{proof}
    \begin{align*}
        \frac{d}{dx}\tan x&= \frac{d}{dx}\left(\frac{\sin x}{\cos x}\right)
        =\frac{\cfrac{d}{dx}\left(\sin x\right)\cos x
        -\sin x\cfrac{d}{dx}\left(\cos x\right)}{\cos^2 x}
        =\frac{\cos x\cos x-\sin x(-\sin x)}{\cos^2 x}  \\
        &= \frac{\sin^2 x+\cos^2 x}{\cos^2 x}=\frac{1}{\cos^2 x}
        =\sec^2 x\qedhere
    \end{align*}
\end{proof}

\subsubsection{Chain Rule}
We have the \textbf{chain rule} formulated by \textbf{James Gregory}
(1638--1675) to find the derivative of a composite function.
\begin{theorem}[Chain Rule]
    If \(f\) and \(g\) are differentiable functions and \(F=f(g(x))\),
    then \(F\) is differentiable and \(F'\) is
    \[F'(x)=f'(g(x))\cdot g'(x)\]
    If \(y=f(u)\) and \(u=g(x)\), then
    \[\frac{dy}{dx}=\frac{dy}{du}\frac{du}{dx}\qedhere\]
\end{theorem}
\begin{proof}
    We know that by definition if \(y=f(x)\),
    then \(\Delta y=f(a+\Delta x)-f(a)\) and
    \[\lim_{\Delta x\to 0}\frac{\Delta y}{\Delta x}=f'(a)\]
    Let \(\epsilon\) be the difference between the difference quotient and the
    derivative, then we have
    \[\lim_{\Delta x\to 0}\epsilon
    =\lim_{\Delta x\to 0}\left(\frac{\Delta y}{\Delta x}-f'(a)\right)
    =f'(a)-f'(a)=0\]
    Thus for a differentiable function \(f\), if we define \(\epsilon=0\)
    when \(\Delta x=0\), then \[\Delta y=f'(a)\Delta x+\epsilon\Delta x\]
    where \(\epsilon\to 0\) as \(\Delta x\to 0\) and \(\epsilon\) is a
    continuous function of \(\Delta x\).
    Suppose that \(u=g(x)\) is differentiable at \(a\) and \(y=f(u)\) is
    differentiable at \(b=g(a)\).
    Then we have
    \[\Delta u=g'(a)\Delta x+\epsilon_1\Delta x=[g'(a)+\epsilon_1]\Delta x\]
    where \(\epsilon_1\to 0\) as \(\Delta x\to 0\).
    Similarly, \[\Delta y=f'(b)\Delta u+\epsilon_2\Delta u=[f'(b)+\epsilon_2]\Delta u\]
    where \(\epsilon_2\to 0\) as \(\Delta u\to 0\).
    We substitute the expression for \(\Delta u\) then we have
    \begin{align*}
        \Delta y &= [f'(b)+\epsilon_2][g'(a)+\epsilon_1]\Delta x \\
        \frac{\Delta y}{\Delta x} &= [f'(b)+\epsilon_2][g'(a)+\epsilon_1]
    \end{align*}
    Since \(\Delta u\to 0\) as \(\Delta x\to 0\), then \(\epsilon_1\to 0\) and
    \(\epsilon_2 \to 0\) as \(\Delta x\to 0\).
    Therefore,
    \[\frac{dy}{dx}=\lim_{\Delta x\to 0}\frac{\Delta y}{\Delta x}
    =\lim_{\Delta x\to 0}[f'(b)+\epsilon_2][g'(a)+\epsilon_1]
    =f'(b)g'(a)=f'(g(a))g'(a)\]
    thus the chain rule is proved.
\end{proof}