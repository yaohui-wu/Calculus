\subsection{Derivatives of Inverse Functions}
\subsubsection{Differentiation and Inverse Functions}
\begin{theorem}
    If \(f\) is a one-to-one continuous function defined on an interval,
    then its inverse function \(f^{-1}\) is also continuous.
\end{theorem}
\begin{theorem}
    If is a one-to-one differentiable function with inverse function \(f^{-1}\)
    and \(f'(f^{-1}(a))\neq 0\), then the inverse function is differentiable at \(a\) and
    \[(f^{-1})'(x)=\frac{1}{f'(f^{-1}(x))}\iff\frac{dy}{dx}
    =\frac{1}{\cfrac{dx}{dy}}\qedhere\]
\end{theorem}

\subsubsection{Logarithmic and Exponential Functions}
The \textbf{Euler's number} \(e\) named after \textbf{Leonhard Euler} (1707--1783)
is the base of the natural exponential function \(y=e^x\).
\begin{definition}[Euler's Number]
    The Euler's number \(e\) is \[e=\lim_{x\to 0}(1+x)^{1/x}\qedhere\]
\end{definition}
Note that the approximate value of \(e\) is \(e\approx2.71828\).
\begin{theorem}
    The exponential function \(f(x)=\log_a x\) is differentiable and
    \[\frac{d}{dx}\log_a x=\frac{1}{x}\log_a e=\frac{1}{x\ln a}\qedhere\]
\end{theorem}
\begin{theorem}
    The derivative of the natural logarithmic function \(f(x)=\ln x\) is
    \[\frac{d}{dx}\ln x=\frac{1}{x}\qedhere\]
\end{theorem}
\begin{problem}
    Prove the power rule \[\frac{d}{dx}x^n=nx^{n-1}\] for \(n\in\R\).
\end{problem}
\begin{solution}
    Let \(y=x^n\), we use implicit and logarithmic differentiation then we have
    \begin{align*}
        \ln y&=\ln x^n\\\ln y&=n\ln x\\\frac{d}{dx}\ln y&=\frac{d}{dx}(n\ln x)
        \\\frac{1}{y}\frac{dy}{dx}&=\frac{n}{x}\\\frac{dy}{dx}&=nx^{-1}y\\
        \frac{dy}{dx}&=nx^{-1}x^n\\\frac{dy}{dx}&=nx^{n-1}\qedhere
    \end{align*}
\end{solution}
\begin{theorem}
    The exponential function \(f(x)=a^x,a>0\) is differentiable and
    \[\frac{d}{dx}a^x=a^x\ln a\qedhere\]
\end{theorem}
\begin{theorem}
    The derivative of the natural exponential function \(f(x)=e^x\) is
    \[\frac{d}{dx}e^x=e^x\qedhere\]
\end{theorem}

\subsubsection{Inverse Trigonometric Functions}
\begin{theorem}
    \[\frac{d}{dx}\arcsin x=\frac{1}{\sqrt{1-x^2}},-1<x<1\qedhere\]
\end{theorem}
\begin{theorem}
    \[\frac{d}{dx}\arccos x=-\frac{1}{\sqrt{1-x^2}},-1<x<1\qedhere\]
\end{theorem}
\begin{theorem}
    \[\frac{d}{dx}\arctan x=\frac{1}{1+x^2}\qedhere\]
\end{theorem}