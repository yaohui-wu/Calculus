\subsection{Derivatives of Inverse Functions}
\subsubsection{Differentiation and Inverse Functions}
\begin{theorem}
    If \(f\) is a one-to-one continuous function defined on an interval,
    then its inverse function \(f^{-1}\) is also continuous.
\end{theorem}
\begin{theorem}
    If \(f\) is a one-to-one differentiable function with inverse function
    \(f^{-1}\) and \(f'(f^{-1}(a))\neq 0\), then the inverse function is
    differentiable at \(a\) and
    \[(f^{-1})'(x)=\frac{1}{f'(f^{-1}(x))}\iff\frac{dy}{dx}
    =\frac{1}{\cfrac{dx}{dy}}\qedhere\]
\end{theorem}

\subsubsection{Derivatives of Logarithmic Functions}
The \textbf{Euler's number} \(e\) named after \textbf{Leonhard Euler} (1707--1783)
is the base of the natural exponential function \(y=e^x\) and the natural
logarithmic function \(y=\ln x\).
\begin{definition}[Euler's Number]
    The Euler's number \(e\) is defined as
    \[e=\lim_{x\to 0}(1+x)^{1/x}
    =\lim_{n\to\infty}\left(1+\frac{1}{n}\right)^n\qedhere\]
\end{definition}
Note that the approximate value of \(e\) is \(e\approx2.71828\).
\begin{theorem}
    The exponential function \(f(x)=\log_a x\) is differentiable and
    \[\frac{d}{dx}\log_a x=\frac{1}{x}\log_a e=\frac{1}{x\ln a}\qedhere\]
\end{theorem}
\begin{proof}
    First we have
    \begin{align*}
        \frac{d}{dx}\log_a x &= \lim_{h\to 0}\frac{\log_a(x+h)-\log_a x}{h}
        =\lim_{h\to 0}\frac{\log_a\left(\dfrac{x+h}{x}\right)}{h}
        =\lim_{h\to 0}\left[\frac{1}{x}\frac{x}{h}
        \log_a\left(1+\frac{h}{x}\right)\right] \\
        &= \frac{1}{x}\left[
            \lim_{h\to 0}\log_a\left(1+\frac{h}{x}\right)^{x/h}\right]
        =\frac{1}{x}\left[
            \lim_{h\to 0}\log_a\left(1+\frac{h}{x}\right)^{1/(h/x)}\right]
        =\frac{1}{x}\log_a e
    \end{align*}
    We know from the change of base formula that
    \[\log_b x=\frac{\log_a x}{\log_a b}\]
    Therefore,
    \[\frac{d}{dx}\log_a x=\frac{1}{x}\log_a e=\frac{1}{x\ln a}\qedhere\]
\end{proof}
\begin{theorem}
    The derivative of the natural logarithmic function \(f(x)=\ln x\) is
    \[\frac{d}{dx}\ln x=\frac{1}{x}\qedhere\]
\end{theorem}
\begin{proof}
    \[\frac{d}{dx}\ln x=\frac{d}{dx}\log_e x=\frac{1}{x\ln e}
    =\frac{1}{x}\qedhere\]
\end{proof}
\begin{problem}
    Find \(f'(x)\) if \(f(x)=\ln|x|\).
\end{problem}
\begin{solution}
    Since \(f(x)=\ln x\) for \(x>0\) and \(f(x)=\ln(-x)\) for \(x<0\),
    it follows that
    \begin{align*}
        f'(x) &= \frac{d}{dx}\ln x=\frac{1}{x},\quad x>0 \\
        f'(x) &= \frac{d}{dx}\ln(-x)=\frac{1}{-x}(-1)=\frac{1}{x},\quad x<0
    \end{align*}
    Therefore, \(\dfrac{d}{dx}\ln|x|=\dfrac{1}{x}\) for all \(x\neq 0\).
\end{solution}

\subsubsection{Derivatives of Exponential Functions}
\begin{theorem}
    The exponential function \(f(x)=a^x,a>0\) is differentiable and
    \[\frac{d}{dx}a^x=a^x\ln a\qedhere\]
\end{theorem}
\begin{proof}
    Let \(y=a^x\iff \log_a y=x\) then by implicit differentiation we have
    \begin{align*}
        \log_a y &= x \\ \frac{1}{y\ln a}\frac{dy}{dx} &= 1 \\
        \frac{dy}{dx} &= y\ln a=a^x\ln a \qedhere
    \end{align*}
\end{proof}
\begin{theorem}
    The derivative of the natural exponential function \(f(x)=e^x\) is
    \[\frac{d}{dx}e^x=e^x\qedhere\]
\end{theorem}
\begin{proof}
    \[\frac{d}{dx}e^x=e^x\ln e=e^x\qedhere\]
\end{proof}
\begin{problem}
    Prove the power rule \(\dfrac{d}{dx}x^n=nx^{n-1}\) for \(n\in\R\).
\end{problem}
\begin{solution}
    The rule is true when \(x=0\) which is trivial so it remains to prove the
    cases for \(x\neq 0\).
    Let \(y=x^n\) such that \(y>0\), by implicit differentiation we have
    \begin{align*}
        \ln y &= \ln x^n=n\ln x \\ \frac{1}{y}\frac{dy}{dx} &= \frac{n}{x} \\
        \frac{dy}{dx} &= nx^{-1}y=nx^{-1}x^n=nx^{n-1}
    \end{align*}
    Similarly, let \(y=x^n\) such that \(y<0\) then we have
    \begin{align*}
        \ln|y| &= \ln(-x^n)=\ln[((-1)^{1/n}x)^n]=n\ln[(-1)^{1/n}x] \\
        \frac{1}{y}\frac{dy}{dx} &= n\frac{1}{(-1)^{1/n}x}(-1)^{1/n}
        =\frac{n}{x} \\ \frac{dy}{dx} &= nx^{-1}y=nx^{-1}x^n=nx^{n-1} \qedhere
    \end{align*}
\end{solution}

\subsubsection{Inverse Trigonometric Functions}
\begin{theorem}
    \[\frac{d}{dx}\arcsin x=\frac{1}{\sqrt{1-x^2}},\quad -1<x<1 \qedhere\]
\end{theorem}
\begin{proof}
    Let \(y=\arcsin x \iff \sin y=x\) and \(-\pi/2\leq y\leq\pi/2\) such that
    \(-1\leq x\leq 1\).
    Note the approximate value of \(\pi\) is \(\pi\approx 3.14159\).
    Then we have
    \begin{align*}
        \frac{d}{dx}\sin y &= \frac{d}{dx}x \\ \cos y\frac{dy}{dx} &= 1 \\
        \frac{dy}{dx} &= \frac{1}{\cos y}
    \end{align*}
    Since \(-\pi/2\leq y\leq\pi/2\) so \(\cos y\geq0\) thus
    \(\cos y=\sqrt{1-\sin^2 y}\).
    Therefore, \(\dfrac{dy}{dx}=\dfrac{1}{\sqrt{1-x^2}}\).
\end{proof}
\begin{theorem}
    \[\frac{d}{dx}\arccos x=-\frac{1}{\sqrt{1-x^2}},\quad -1<x<1 \qedhere\]
\end{theorem}
\begin{proof}
    Let \(y=\arccos x \iff \cos y=x\) and \(0\leq y\leq\pi\) such that
    \(-1\leq x\leq 1\).
    Then we have
    \begin{align*}
        \frac{d}{dx}\cos y &= \frac{d}{dx}x \\ -\sin y\frac{dy}{dx} &= 1 \\
        \frac{dy}{dx} &= -\frac{1}{\sin y}
    \end{align*}
    Since \(0\leq y\leq\pi\) so \(\sin y\geq0\) thus
    \(\sin y=\sqrt{1-\cos^2 y}\).
    Therefore, \(\dfrac{dy}{dx}=-\dfrac{1}{\sqrt{1-x^2}}\).
\end{proof}
\begin{theorem}
    \[\frac{d}{dx}\arctan x=\frac{1}{1+x^2} \qedhere\]
\end{theorem}
\begin{proof}
    Let \(y=\arctan x \iff \tan y=x\).
    Then we have
    \begin{align*}
        \frac{d}{dx}\tan y &= \frac{d}{dx}x \\ \sec^2 y\frac{dy}{dx} &= 1 \\
        \frac{dy}{dx} &= \cos^2 y
    \end{align*}
    Then we can show that
    \begin{align*}
        \tan y &= x \\ 1+\frac{\sin^2 y}{\cos^2 y} &= 1+x^2\\
        1+\frac{1-\cos^2 y}{\cos^2 y} &= 1+x^2 \\ \sec^2 y &= 1+x^2 \\
        \cos^2y &= \frac{1}{1+x^2}
    \end{align*}
    Therefore, \(\dfrac{dy}{dx}=\dfrac{1}{1+x^2}\).
\end{proof}