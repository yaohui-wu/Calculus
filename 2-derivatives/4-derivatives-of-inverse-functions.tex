\subsection{Derivatives of Inverse Functions}
\subsubsection{Differentiation and Inverse Functions}
\begin{theorem}
    If \(f\) is a one-to-one continuous function defined on an interval,
    then its inverse function \(f^{-1}\) is also continuous.
\end{theorem}
\begin{theorem}
    If is a one-to-one differentiable function with inverse function \(f^{-1}\)
    and \(f'(f^{-1}(a))\neq 0\), then the inverse function is differentiable at \(a\) and
    \[(f^{-1})'(x)=\frac{1}{f'(f^{-1}(x))}\iff\frac{dy}{dx}
    =\frac{1}{\cfrac{dx}{dy}}\qedhere\]
\end{theorem}

\subsubsection{Logarithmic and Exponential Functions}
The \textbf{Euler's number} \(e\) named after \textbf{Leonhard Euler} (1707--1783)
is the base of the natural exponential function \(y=e^x\).
\begin{definition}[Euler's Number]
    The Euler's number \(e\) is \[e=\lim_{x\to 0}(1+x)^{1/x}\qedhere\]
\end{definition}
Note that the approximate value of \(e\) is \(e\approx2.71828\).
\begin{theorem}
    The exponential function \(f(x)=\log_a x\) is differentiable and
    \[\frac{d}{dx}\log_a x=\frac{1}{x}\log_a e=\frac{1}{x\ln a}\qedhere\]
\end{theorem}
\begin{theorem}
    The derivative of the natural logarithmic function \(f(x)=\ln x\) is
    \[\frac{d}{dx}\ln x=\frac{1}{x}\qedhere\]
\end{theorem}
\begin{theorem}
    The exponential function \(f(x)=a^x,a>0\) is differentiable and
    \[\frac{d}{dx}a^x=a^x\ln a\qedhere\]
\end{theorem}
\begin{theorem}
    The derivative of the natural exponential function \(f(x)=e^x\) is
    \[\frac{d}{dx}e^x=e^x\qedhere\]
\end{theorem}
\begin{problem}
    Prove the power rule \(\dfrac{d}{dx}x^n=nx^{n-1}\) for \(n\in\R\).
\end{problem}
\begin{solution}
    The rule is true when \(x=0\) which is trivial so it remains to prove the
    cases for \(x\neq 0\).
    Let \(y=x^n\) and \(y>0\), we use implicit differentiation and logarithmic
    differentiation then we have
    \begin{align*}
        \ln y &= \ln x^n \\ \ln y &= n\ln x \\
        \frac{d}{dx}\ln y &= \frac{d}{dx}n\ln x \\
        \frac{1}{y}\frac{dy}{dx} &= \frac{n}{x} \\
        \frac{dy}{dx} &= nx^{-1}y \\ \frac{dy}{dx} &= nx^{-1}x^n \\
        \frac{dy}{dx} &= nx^{n-1}
    \end{align*}
    Similarly, we can prove that the rule is true for \(x^n<0\).
\end{solution}

\subsubsection{Inverse Trigonometric Functions}
\begin{theorem}
    \[\frac{d}{dx}\arcsin x=\frac{1}{\sqrt{1-x^2}},\quad -1<x<1 \qedhere\]
\end{theorem}
\begin{proof}
    Let \(y=\arcsin x \iff \sin y=x\) and \(-\pi/2\leq y\leq\pi/2\) such that
    \(-1\leq x\leq 1\).
    Note the approximate value of \(\pi\) is \(\pi\approx 3.14159\).
    Then we have
    \begin{align*}
        \frac{d}{dx}\sin y &= \frac{d}{dx}x \\ \cos y\frac{dy}{dx} &= 1 \\
        \frac{dy}{dx} &= \frac{1}{\cos y}
    \end{align*}
    Since \(-\pi/2\leq y\leq\pi/2\) so \(\cos y\geq0\) thus
    \(\cos y=\sqrt{1-\sin^2 y}\).
    Therefore, \(\dfrac{dy}{dx}=\dfrac{1}{\sqrt{1-x^2}}\).
\end{proof}
\begin{theorem}
    \[\frac{d}{dx}\arccos x=-\frac{1}{\sqrt{1-x^2}},\quad -1<x<1 \qedhere\]
\end{theorem}
\begin{proof}
    Let \(y=\arccos x \iff \cos y=x\) and \(0\leq y\leq\pi\) such that
    \(-1\leq x\leq 1\).
    Then we have
    \begin{align*}
        \frac{d}{dx}\cos y &= \frac{d}{dx}x \\ -\sin y\frac{dy}{dx} &= 1 \\
        \frac{dy}{dx} &= -\frac{1}{\sin y}
    \end{align*}
    Since \(0\leq y\leq\pi\) so \(\sin y\geq0\) thus
    \(\sin y=\sqrt{1-\cos^2 y}\).
    Therefore, \(\dfrac{dy}{dx}=-\dfrac{1}{\sqrt{1-x^2}}\).
\end{proof}
\begin{theorem}
    \[\frac{d}{dx}\arctan x=\frac{1}{1+x^2} \qedhere\]
\end{theorem}
\begin{proof}
    Let \(y=\arctan x \iff \tan y=x\).
    Then we have
    \begin{align*}
        \frac{d}{dx}\tan y &= \frac{d}{dx}x \\ \sec^2 y\frac{dy}{dx} &= 1 \\
        \frac{dy}{dx} &= \cos^2 y
    \end{align*}
    Then we can show that
    \begin{align*}
        \tan y &= x \\ 1+\frac{\sin^2 y}{\cos^2 y} &= 1+x^2\\
        1+\frac{1-\cos^2 y}{\cos^2 y} &= 1+x^2 \\ \sec^2 y &= 1+x^2 \\
        \cos^2y &= \frac{1}{1+x^2}
    \end{align*}
    Therefore, \(\dfrac{dy}{dx}=\dfrac{1}{1+x^2}\).
\end{proof}