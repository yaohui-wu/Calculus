\subsection{The Mean Value Theorem}
\textbf{Michel Rolle} (1652--1719)
\begin{theorem}[Rolle's Theorem]
    Suppose that \(f\) is a continuous function on the closed interval \([a,b]\),
    \(f\) is differentiable on the open interval \((a,b)\), and \(f(a)=f(b)\).
    Then there is a number \(c\) in \((a,b)\) such that \(f'(c)=0\).
\end{theorem}
\textbf{Joseph-Louis Lagrange} (1736--1813)
\begin{theorem}[Lagrange's Mean Value Theorem]
    Suppose that \(f\) is a continuous function on the closed interval \([a,b]\),
    \(f\) is differentiable on the open interval \((a,b)\), and \(f(a)=f(b)\).
    Then there is a number \(c\) in \((a,b)\) such that
    \[f'(c)=\frac{f(b)-f(a)}{b-a}\qedhere\]
\end{theorem}
\begin{theorem}[Cauchy's Mean Value Theorem]
    Suppose that the fucntions \(f\) and \(g\) are continuous on \([a,b]\) and
    differentiable on \((a,b)\) and \(g(x)\neq 0\) for all \(x\) in \((a,b)\).
    Then there is a number \(c\) in \(a,b\) such that
    \[\frac{f'(c)}{g'(c)}=\frac{f(b)-f(a)}{g(b)-g(a)}\qedhere\]
\end{theorem}