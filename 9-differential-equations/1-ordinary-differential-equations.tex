\subsection{Ordinary Differential Equations}
A \textbf{differential equation} is an equation that relates some unknown
functions and their derivatives.
An \textbf{ordinary differential equation} (ODE) is a differential equation
that relates one or more functions of a single variable and their ordinary
derivatives.
The \textbf{order} of a differential equation is the highest order of the
derivative in the equation.
Newton's second law of motion \(F=ma\) is an ordinary differential
equation since we can write it in the form
\[F=m\frac{dv}{dt}\]
which is a first order differential equation, or
\[F=m\frac{d^2s}{dt^2}\]
 which is a second order differential equation.
A function \(f\) is a \textbf{solution} of a differential equation if the
function and its derivatives satisfy the equation for all values of \(x\) in
some open interval \(a<x<b\).
It is possible that there are many solutions of a differential equation.
An \textbf{initial condition} is a condition \(y(x_0)=y_0\) or
\(y^{(n)}(x_0)=y_n\) on the solution.
An \textbf{initial value problem} is solving a differential equation with initial conditions.
The \textbf{interval of validity} is the largest possible interval on which
the solution is valid and contains \(x_0\) in the initial conditions.
The general solution of a differential equation is a family of solutions in
the most general form.
The particular solution is the solution that satisfies the initial conditions.
An explicit solution is any solution in the form \(y=y(x)\), otherwise it is
an implicit solution.
The particular solution of the differential equation
\[\frac{dy}{dx}=y\]
with initial
condition \(y(0)=1\) is \(y=e^x\) since
\[\frac{dy}{dx}=\frac{d}{dx}e^x=e^x=y\]
and \(y(0)=e^0=1\).
The \textbf{existence} and \textbf{uniqueness} problem asks that given a
differential equation, does there exist a solution and if any is there only
one solution.