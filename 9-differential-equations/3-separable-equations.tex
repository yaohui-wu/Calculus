\subsection{Separable Equations}
\subsubsection{Separation of Variables}
A \textbf{separable equation} is a first order differential equation that can
be written in the differential form
\[M(x)\,dx+N(y)\,dy=0\]
or can be written in the form
\[\frac{dy}{dx}=f(x)g(y)\]
We can separate the variables if \(g(y)\neq0\), let \(h(y)=1/g(y)\) then
\[\frac{dy}{dx}=\frac{f(x)}{h(y)}\]
We write the equation in the differential form
\[h(y)\,dy=f(x)\,dx\]
Then we integrate both sides of the equation:
\[\int h(y)\,dy=\int f(x)\,dx\]
Now we have an implicit solution of the differential equation and sometimes we
can solve for an explicit solution.
We can justify the method of separation of variables by using the chain rule
to show that
\begin{align*}
    \frac{d}{dx}\int h(y)\,dy &= \frac{d}{dx}\int f(x)\,dx \\
    \frac{d}{dy}\int h(y)\,dy\frac{dy}{dx} &= f(x) \\
    h(y)\frac{dy}{dx} &= f(x) \\
    \frac{dy}{dx} &= \frac{f(x)}{h(y)}=f(x)g(y)
\end{align*}
\begin{problem}
    Solve the differential equation \(\dfrac{dy}{dx}=-xy\).
\end{problem}
\begin{solution}
    Notice that \(y=0\) is a trivial solution then we solve the differential
    equation for non-trivial solutions \(y\neq 0\).
    We use separation of variables then
    \begin{align*}
        \dfrac{dy}{dx} &= -xy \\
        \int\frac{dy}{y} &= -\int x\,dx \\
        \ln |y|+C_1 &= -\frac{x^2}{2}+C_2
    \end{align*}
    Let \(C=C_2-C_1\), then
    \begin{align*}
        \ln |y| &= -\frac{x^2}{2}+C_2-C_1=-\frac{x^2}{2}+C \\
        |y| &= e^{(-x^2/2)+C}=e^Ce^{-x^2/2} \\
        y &= \pm e^Ce^{-x^2/2}=Ae^{-x^2/2}
    \end{align*}
    where \(A\in\R\) is an arbitrary constant.
\end{solution}
\begin{problem}
    Solve the differential equation \(\dfrac{dy}{dx}=\dfrac{x^2}{y^2}\) with
    the initial condition \(y(0)=2\).
\end{problem}
\begin{solution}
    We use separation of variables to find the general solution then
    \begin{align*}
        \dfrac{dy}{dx} &= \dfrac{x^2}{y^2} \\
        \int y^2\,dy &= \int x^2\,dx \\
        \frac{y^3}{3}+C_1 &= \frac{x^3}{3}+C_2 \\
        \frac{y^3}{3} &=\frac{x^3}{3}+C,\quad C=C_2-C_1 \\
        y^3 &= x^3+3C \\
        y &= \sqrt[3]{x^3+K},\quad K=3C
    \end{align*}
    We consider the initial condition \(y(0)=2\) to find the particular
    solution then
    \[y(0)=\sqrt[3]{0+K}\iff 2=\sqrt[3]{K}\iff K=8\]
    The solution of the initial value problem is \(y=\sqrt[3]{x^3+8}\).
\end{solution}
\begin{problem}
    Solve the differential equation \(\dfrac{dy}{dx}=\dfrac{6x^2}{2y+\cos y}\).
\end{problem}
\begin{solution}
    We have
    \begin{align*}
        \dfrac{dy}{dx} &= \dfrac{6x^2}{2y+\cos y} \\
        \int(2y+\cos y)\,dy &= \int 6x^2\,dx \\
        y^2+\sin y &= 2x^3+C
    \end{align*}
    where \(C\in\R\) is an arbitrary constant.
\end{solution}
\begin{problem}
    Solve the differential equation \(y'=x^2y\).
\end{problem}
\begin{solution}
    We have
    \begin{align*}
        \frac{dy}{dx} &= x^2y \\ \int \frac{dy}{y} &= \int x^2\,dx \\
        \ln |y| &= \frac{x^3}{3}+C \\ |y| &= e^{(x^3/3)+C}\\ y &= Ae^{x^3/3}
    \end{align*}
    where \(A\in\R\) is an arbitrary constant.
\end{solution}

\subsubsection{Homogeneous Equations}
A \textbf{homogeneous} equation is in the form
\[\frac{dy}{dx}=f\left(\frac{y}{x}\right)\]
We can transform a homogeneous equation into a separable equation by a change
of variable.
Consider the equation
\[\frac{dy}{dx}=\frac{y-4x}{x-y}\]
and we can show that
\[\frac{dy}{dx}=\frac{(y/x)-4}{1-(y/x)}\]
thus the equation is homogeneous.
Let \(v=y/x\iff y=vx\) so
\begin{align*}
    \frac{dy}{dx}=\frac{dv}{dx}x+v &= \frac{v-4}{1-v} \\
    \frac{dv}{dx}x
    &= \frac{v-4}{1-v}-v=\frac{v-4-v(1-v)}{1-v}=\frac{v^2-4}{1-v}
\end{align*}
and the equation is separable then
\begin{align*}
    \int\frac{1-v}{v^2-4}\,dv &= \int\frac{dx}{x}
\end{align*}
Since
\[\int\frac{1-v}{v^2-4}\,dv=\int\frac{dv}{v^2-4}-\int\frac{v}{v^2-4}\,dv\]
then
\begin{align*}
    \frac{1}{v^2-4}=\frac{1}{(v+2)(v-2)} &= \frac{A}{v+2}+\frac{B}{v-2} \\
    1 &= A(v-2)+B(v+2) \\
    v=-2\iff -4A=1\iff A &= -\frac{1}{4} \\
    v=2\iff 4B=1\iff B &= \frac{1}{4}
\end{align*}
so
\[\int\frac{dv}{v^2-4}
=-\frac{1}{4}\int\frac{dv}{v+2}+\frac{1}{4}\int\frac{dv}{v-2}
=-\frac{1}{4}\ln|v+2|+\frac{1}{4}\ln|v-2|\]
and
\[\int\frac{v}{v^2-4}\,dv=\frac{1}{2}\ln|v^2-4|\]
therefore
\[\int\frac{1-v}{v^2-4}\,dv
=-\frac{1}{4}\ln|v+2|+\frac{1}{4}\ln|v-2|-\frac{1}{2}\ln|v^2-4|\]
Then
\begin{align*}
    -\frac{1}{4}\ln|v+2|+\frac{1}{4}\ln|v-2|-\frac{1}{2}\ln|v^2-4|
    &= \ln|x|+C_1 \\
    \ln|v-2|-\ln|v+2|-2\ln|v^2-4| &= 4\ln|x|+C_2,\quad C_2=4C_1 \\
    \ln\left|\frac{v-2}{v+2}\right|-\ln((v^2-4)^2)
    =\ln\left|\frac{v-2}{(v+2)(v+2)^2(v-2)^2}\right| &= \ln(x^4)+C_2 \\
    \ln\left|\frac{1}{(v+2)^3(v-2)}\right| &= \ln(x^4)+C_2 \\
    \frac{1}{(v+2)^3(v-2)} &= C_3x^4,\quad C_3=\pm e^{C_2} \\
    (v+2)^3(v-2)x^4 &= C,\quad C=1/C_3 \\
    (vx+2x)^3(vx-2x) &= C
\end{align*}
Therefore the solution is
\[(y+2x)^3(y-2x)=C\]
\begin{problem}
    Solve the differential equation
    \[\frac{dy}{dx}=\frac{x^2+xy+y^2}{x^2}\]
\end{problem}
\begin{solution}
    Since
    \[\frac{dy}{dx}=\frac{x^2+xy+y^2}{x^2}
    =1+\frac{y}{x}+\left(\frac{y}{x}\right)^2\]
    hence the equation is homogeneous.
    Let \(y=vx\), then
    \begin{align*}
        \frac{dy}{dx}=\frac{dv}{dx}x+v &= 1+v+v^2 \\
        \frac{dv}{dx}x &= 1+v^2
    \end{align*}
    and by separation of variables
    \begin{align*}
        \int\frac{dv}{1+v^2} &= \int\frac{dx}{x} \\
        \arctan v &= \ln x+C \\
        \arctan\left(\frac{y}{x}\right)-\ln x &= C \qedhere
    \end{align*}
\end{solution}

\subsubsection{Mathematical Modeling}
\begin{problem}
    According to Newton's law of universal gravitation,
    the gravitational force on an object of mass \(m\) that has been
    projected vertically upward from the earth's surface is
    \[F=\frac{mgR^2}{(x+R)^2}\]
    where \(x=x(t)\) is the object's distance
    above the surface at time \(t\), \(R\) is the
    earth's radius, and \(g\) is the acceleration due to gravity.
    Also, by Newton's second law,
    \[F=ma=m\frac{dv}{dt}=-\frac{mgR^2}{(x+R)^2}\]
    Suppose a rocket is fired vertically upward with an initial velocity \(v_0\).
    Let \(h\) be the maximum height above the surface reached by the object.
    Show that \(v_0=\sqrt{\dfrac{2gRh}{R+h}}\) and compute
    \(v_e=\lim_{h\to\infty}v_0\), the escape velocity of the earth, using
    \(R=6,378\) km and \(g=9.8\ \text{m/s}^2\).
\end{problem}
\begin{solution}
    By the chain rule
    \[\frac{dv}{dt}=\frac{dx}{dt}\frac{dv}{dx}=v\frac{dv}{dx}\]
    then
    \begin{align*}
        m\frac{dv}{dt}=mv\frac{dv}{dx} &= -\frac{mgR^2}{(x+R)^2} \\
        v\,dv &= -\frac{gR^2}{(x+R)^2}\,dx
    \end{align*}
    Since the height is zero when \(v=v_0\) and the height is maximum when
    \(v=0\) then
    \begin{align*}
        \int_{v_0}^0 v\,dv &= -\int_0^h \frac{gR^2}{(x+R)^2}\,dx \\
        \left[\frac{v^2}{2}\right]_{v_0}^0
        &= \left[\frac{gR^2}{x+R}\right]_0^h \\
        -\frac{(v_0)^2}{2} &= gR^2\left(\frac{1}{R+h}-\frac{1}{R}\right)
        =gR^2\left(\frac{R-(R+h)}{R(R+h)}\right)=-\frac{gRh}{R+h} \\
        v_0= &= \sqrt{\dfrac{2gRh}{R+h}}
    \end{align*}
    The escape velocity of the earth is
    \begin{align*}
        v_e &= \lim_{h\to\infty}\sqrt{\dfrac{2gRh}{R+h}}
        =\sqrt{\lim_{h\to\infty}\frac{2gR}{(R/h)+1}}=\sqrt{\frac{2gR}{0+1}}
        =\sqrt{2gR} \\
        &=\sqrt{2(9.8)(6.378\times10^6)}\ \text{m/s}
        \approx 11.1807\ \text{km/s} \qedhere
    \end{align*}
\end{solution}