\subsection{Separable Equations}
\subsubsection{Separation of Variables}
A \textbf{separable equation} is a first order differential equation that can
be written in the form \[\frac{dy}{dx}=f(x)g(y)\]
We can separate the variables if \(g(y)\neq0\), let \(h(y)=1/g(y)\) then we
have \[\frac{dy}{dx}=\frac{f(x)}{h(y)}\]
We write the equation in the differential form \[h(y)\,dy=f(x)\,dx\] so that
there is a function of \(y\) on one side of the equation and a function of
\(x\) on the other side.
Then we integrate both sides of the equation with respect to the variable of
the function.
\[\int h(y)\,dy=\int f(x)\,dx\]
Now we have an implicit solution and in some cases we can solve for an
explicit solution.
We use the chain rule then we have
\begin{align*}
    \frac{d}{dx}\int h(y)\,dy &= \frac{d}{dx}\int f(x)\,dx \\
    \frac{d}{dy}\int h(y)\,dy\frac{dy}{dx} &= f(x) \\
    h(y)\frac{dy}{dx} &= f(x)
\end{align*}
Therefore, we solved the differential equation by
\textbf{separation of variables}.
\begin{problem}
    Solve the differential equation \(\dfrac{dy}{dx}=-xy\).
\end{problem}
\begin{solution}
    Notice that \(y=0\) is a trivial solution then we solve the differential
    equation assuming that \(y\neq 0\).
    We use separation of variables then we have
    \begin{align*}
        \dfrac{dy}{dx}&=-xy\\\frac{dy}{y}&=-x\,dx\\\int\frac{dy}{y}
        &=-\int x\,dx\\\ln |y|+C_1&=-\frac{x^2}{2}+C_2
    \end{align*}
    Let \(C=C_2-C_1\) where \(C\in\R\), then we have
    \begin{align*}
        \ln |y|&=-\frac{x^2}{2}+C\\e^{\ln |y|}&=e^{-(x^2/2)+C}\\
        |y|&=e^Ce^{-x^2/2}\\y&=Ae^{-x^2/2}
    \end{align*}
    where \(A\in\R\) is an arbitrary constant.
\end{solution}
\begin{problem}
    Solve the differential equation \(\dfrac{dy}{dx}=\dfrac{x^2}{y^2}\) with
    the initial condition \(y(0)=2\).
\end{problem}
\begin{solution}
    We use separation of variables to find the general solution then we have
    \begin{align*}
        \dfrac{dy}{dx} &= \dfrac{x^2}{y^2} \\ y^2\,dy&= x^2\,dx \\
        \int y^2\,dy &= \int x^2\,dx \\
        \frac{y^3}{3}+C_1 &= \frac{x^3}{3}+C_2 \\
        \frac{y^3}{3} &=\frac{x^3}{3}+C,\quad C=C_2-C_1 \\
        y^3 &= x^3+3C \\ y &= \sqrt[3]{x^3+3C}
    \end{align*}
    We consider the initial condition to find the actual solution then we have
    \[y(0)=2=\sqrt[3]{(0)^3+3C} \implies 2=\sqrt[3]{3C} \implies 3C=8\]
    Thus, the solution to the differential equation is \(y=\sqrt[3]{x^3+8}\).
\end{solution}
\begin{problem}
    Solve the differential equation \(\dfrac{dy}{dx}=\dfrac{6x^2}{2y+\cos y}\).
\end{problem}
\begin{solution}
    We have
    \begin{align*}
        \dfrac{dy}{dx} &= \dfrac{6x^2}{2y+\cos y} \\
        \int(2y+\cos y)\,dy &= \int 6x^2\,dx \\
        y^2+\sin y &= 2x^3+C
    \end{align*}
    where \(C\in\R\) is an arbitrary constant.
\end{solution}
\begin{problem}
    Solve the differential equation \(y'=x^2y\).
\end{problem}
\begin{solution}
    We have
    \begin{align*}
        \frac{dy}{dx} &= x^2y \\ \int \frac{dy}{y} &= \int x^2\,dx \\
        \ln |y| &= \frac{x^3}{3}+C \\ |y| &= e^{(x^3/3)+C}\\ y &= Ae^{x^3/3}
    \end{align*}
    where \(A\in\R\) is an arbitrary constant.
\end{solution}
\begin{problem}
    \textbf{Newton's law of universal gravitation} states that
    \[F=G\frac{m_1m_2}{r^2}\]
    Then the gravitational force on an object of mass \(m\) that has been
    projected vertically upward from the earth's surface is
    \[F=\frac{mgR^2}{(x+R)^2}\] where \(x=x(t)\) is the object's distance
    above the surface at time \(t\), \(R\) is the
    earth's radius, and \(g\) is the acceleration due to gravity.
    Also, by Newton's second law, we have
    \[F=ma=m\frac{dv}{dt}=-\frac{mgR^2}{(x+R)^2}\]
    Suppose a rocket is fired vertically upward with an initial velocity \(v_0\).
    Let \(h\) be the maximum height above the surface reached by the object.
    Show that \(v_0=\sqrt{\dfrac{2gRh}{R+h}}\) and compute \(v_e\), the escape
    velocity of the earth, using \(R=6,378\) km and \(g=9.8\ \text{m/s}^2\).
\end{problem}