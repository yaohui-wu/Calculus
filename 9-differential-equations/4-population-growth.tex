\subsection{Population Growth}
Let \(y=y(t)\) be a function representing the value of a quantity \(y\) at
time \(t\) such that
\[\frac{dy}{dt}=ky\]
where \(k\) is a constant, then the differential equation is the law of
natural growth if \(k>0\) or the law of natural decay if \(k<0\).
Since the differential equation is separable, then
\begin{align*}
    \frac{dy}{dt} &= ky \\
    \int\frac{dy}{y} &= \int k\,dt \\
    \ln|y| &= kt+C \\
    |y| &= e^{kt+C}=e^Ce^{kt} \\
    y &= \pm e^Ce^{kt}=Ae^{kt}
\end{align*}
so \(y=Ae^{kt}\) is the general solution of the differential equation.
It follows that
\[\frac{dy}{dt}=kAe^{kt}=ky\]
and \(y(0)=Ae^0=A\) is the initial value of the function \(y(t)\).

\subsubsection{Logistic Growth}
If \(M\) is the carrying capacity and \(0<y<M\), then the logistic
differential equation is
\[\frac{dy}{dt}=ky(M-y)\]
We use separation of variables then
\begin{align*}
    \frac{dy}{dt} &= ky(M-y) \\
    \int\frac{dy}{y(M-y)} &= \int k\,dt
\end{align*}
and using partial fractions
\begin{align*}
    \frac{1}{y(M-y)} &= \frac{A}{y}+\frac{B}{M-y} \\
    1 &= A(M-y)+By \\
    y=0 &\iff A=\frac{1}{M} \\
    y=M &\iff B=\frac{1}{M}
\end{align*}
therefore
\begin{align*}
    \frac{1}{M}\int\left(\frac{1}{y}+\frac{1}{M-y}\right)\,dy &= \int k\,dt \\
    \frac{1}{M}(\ln|y|-\ln|M-y|) &= kt+C_1 \\
    \ln\frac{y}{M-y} &= kMt+C_2 \\
    \frac{y}{M-y} &= e^{kMt+C_2}=e^{C_2}e^{kMt}=Ae^{kMt}
\end{align*}
If the population at time \(t=0\) is \(y(0)=y_0\),
then \(A=y_0/(M-y_0)\) and so
\begin{align*}
    \frac{y}{M-y} &= \frac{y_0}{M-y_0}e^{kMt} \\
    (M-y_0)y &= y_0e^{kMt}(M-y)=y_0Me^{kMt}-y_0e^{kMt}y \\
    (M-y_0)y+y_0e^{kMt}y=(M-y_0+y_0e^{kMt})y &= y_0Me^{kMt} \\
    y &= \frac{y_0Me^{kMt}}{M-y_0+y_0e^{kMt}}
\end{align*}
then
\[y=\frac{y_0M}{(M-y_0+y_0e^{kMt})e^{-kMt}}=\frac{y_0M}{y_0+(M-y_0)e^{-kMt}}\]
is the solution of the differential equation and
\[\lim_{t\to\infty}y(t)=\frac{y_0M}{y_0+0}=M\]
We can show that
\begin{align*}
    \frac{d^2y}{dt^2} &= \frac{d}{dt}(kMy-ky^2)=kM\frac{dy}{dt}-2ky\frac{dy}{dt}
    =k(M-2y)\frac{dy}{dt}=k(M-2y)ky(M-y) \\
    &= k^2y(M-y)(M-2y)
\end{align*}
Then
\[k^2y(M-y)(M-2y)=0\]
and
\begin{align*}
    k^2y &= 0 & M-y &= 0 & M-2y &= 0 \\
    y &= 0 & y &= M & y &= \frac{M}{2}
\end{align*}
so
\begin{align*}
    &\left.\frac{dy}{dt}\right|_{y=0}=k(0)(M-0)=0 \\
    &\left.\frac{dy}{dt}\right|_{y=M}=kM(M-M)=0 \\
    &\left.\frac{dy}{dt}\right|_{y=M/2}=k\frac{M}{2}\left(M-\frac{M}{2}\right)
    =\frac{kM^2}{4}
\end{align*}
We deduce that a population grows fastest when it reaches half its carrying
capacity.